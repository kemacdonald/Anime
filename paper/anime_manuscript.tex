\documentclass[english,floatsintext,man]{apa6}

\usepackage{amssymb,amsmath}
\usepackage{ifxetex,ifluatex}
\usepackage{fixltx2e} % provides \textsubscript
\ifnum 0\ifxetex 1\fi\ifluatex 1\fi=0 % if pdftex
  \usepackage[T1]{fontenc}
  \usepackage[utf8]{inputenc}
\else % if luatex or xelatex
  \ifxetex
    \usepackage{mathspec}
    \usepackage{xltxtra,xunicode}
  \else
    \usepackage{fontspec}
  \fi
  \defaultfontfeatures{Mapping=tex-text,Scale=MatchLowercase}
  \newcommand{\euro}{€}
\fi
% use upquote if available, for straight quotes in verbatim environments
\IfFileExists{upquote.sty}{\usepackage{upquote}}{}
% use microtype if available
\IfFileExists{microtype.sty}{\usepackage{microtype}}{}

% Table formatting
\usepackage{longtable, booktabs}
\usepackage{lscape}
% \usepackage[counterclockwise]{rotating}   % Landscape page setup for large tables
\usepackage{multirow}		% Table styling
\usepackage{tabularx}		% Control Column width
\usepackage[flushleft]{threeparttable}	% Allows for three part tables with a specified notes section
\usepackage{threeparttablex}            % Lets threeparttable work with longtable

% Create new environments so endfloat can handle them
% \newenvironment{ltable}
%   {\begin{landscape}\begin{center}\begin{threeparttable}}
%   {\end{threeparttable}\end{center}\end{landscape}}

\newenvironment{lltable}
  {\begin{landscape}\begin{center}\begin{ThreePartTable}}
  {\end{ThreePartTable}\end{center}\end{landscape}}




% The following enables adjusting longtable caption width to table width
% Solution found at http://golatex.de/longtable-mit-caption-so-breit-wie-die-tabelle-t15767.html
\makeatletter
\newcommand\LastLTentrywidth{1em}
\newlength\longtablewidth
\setlength{\longtablewidth}{1in}
\newcommand\getlongtablewidth{%
 \begingroup
  \ifcsname LT@\roman{LT@tables}\endcsname
  \global\longtablewidth=0pt
  \renewcommand\LT@entry[2]{\global\advance\longtablewidth by ##2\relax\gdef\LastLTentrywidth{##2}}%
  \@nameuse{LT@\roman{LT@tables}}%
  \fi
\endgroup}


  \usepackage{graphicx}
  \makeatletter
  \def\maxwidth{\ifdim\Gin@nat@width>\linewidth\linewidth\else\Gin@nat@width\fi}
  \def\maxheight{\ifdim\Gin@nat@height>\textheight\textheight\else\Gin@nat@height\fi}
  \makeatother
  % Scale images if necessary, so that they will not overflow the page
  % margins by default, and it is still possible to overwrite the defaults
  % using explicit options in \includegraphics[width, height, ...]{}
  \setkeys{Gin}{width=\maxwidth,height=\maxheight,keepaspectratio}
\ifxetex
  \usepackage[setpagesize=false, % page size defined by xetex
              unicode=false, % unicode breaks when used with xetex
              xetex]{hyperref}
\else
  \usepackage[unicode=true]{hyperref}
\fi
\hypersetup{breaklinks=true,
            pdfauthor={},
            pdftitle={Moos as cues: One-to-one biases in a non-linguistic and non-communicative domain},
            colorlinks=true,
            citecolor=blue,
            urlcolor=blue,
            linkcolor=black,
            pdfborder={0 0 0}}
\urlstyle{same}  % don't use monospace font for urls

\setlength{\parindent}{0pt}
%\setlength{\parskip}{0pt plus 0pt minus 0pt}

\setlength{\emergencystretch}{3em}  % prevent overfull lines

\ifxetex
  \usepackage{polyglossia}
  \setmainlanguage{}
\else
  \usepackage[english]{babel}
\fi

% Manuscript styling
\captionsetup{font=singlespacing,justification=justified}
\usepackage{csquotes}
\usepackage{upgreek}



\usepackage{tikz} % Variable definition to generate author note

% fix for \tightlist problem in pandoc 1.14
\providecommand{\tightlist}{%
  \setlength{\itemsep}{0pt}\setlength{\parskip}{0pt}}

% Essential manuscript parts
  \title{Moos as cues: One-to-one biases in a non-linguistic and
non-communicative domain}

  \shorttitle{Moos as cues}


  \author{Kyle MacDonald\textsuperscript{1}, Ricardo A. H. Bion\textsuperscript{1}, Virginia A. Marchman\textsuperscript{1}, \& Anne Fernald\textsuperscript{1}}

  \def\affdep{{"", "", "", ""}}%
  \def\affcity{{"", "", "", ""}}%

  \affiliation{
    \vspace{0.5cm}
          \textsuperscript{1} Stanford University  }

  \authornote{
    \newcounter{author}
    Enter author note here.

                      Correspondence concerning this article should be addressed to Kyle MacDonald, 450 Serra Mall, Stanford, CA 94306. E-mail: \href{mailto:kylem4@stanford.edu}{\nolinkurl{kylem4@stanford.edu}}
                                              }


  \abstract{When hearing a novel name, children tend to select a novel object rather
than a familiar one, a bias known as disambiguation. This bias is often
assumed to reflect children's expectations about the nature of words or
expectations about the communicative intention of speakers. This study
investigated whether similar biases emerge in a domain that is
non-linguistic and non-communicative for children, but in which strong
regularities can be found: the vocalizations that animals produce. Using
online processing measures, we first show that two-year-olds can
identify familiar animals based on their vocalizations, though not as
fast as they are to identify these same animals when hearing their
names. We then show that children rapidly look at an unfamiliar animal
when hearing a novel animal vocalization or novel animal name, being
equally fast in both conditions. In an additional experiment, we
replicate the finding that children look at a novel animal when hearing
a novel animal vocalization, but show that these biases do not
necessarily lead to learning. We characterize disambiguation biases as
resulting from domain-general processing mechanisms, rather than from
lexical or communicative constraints.}
  \keywords{disambiguation, mutual exclusivity, environmental sounds, retention,
word learning \\

    \indent Word count: X
  }





\usepackage{amsthm}
\newtheorem{theorem}{Theorem}
\newtheorem{lemma}{Lemma}
\theoremstyle{definition}
\newtheorem{definition}{Definition}
\newtheorem{corollary}{Corollary}
\newtheorem{proposition}{Proposition}
\theoremstyle{definition}
\newtheorem{example}{Example}
\theoremstyle{definition}
\newtheorem{exercise}{Exercise}
\theoremstyle{remark}
\newtheorem*{remark}{Remark}
\newtheorem*{solution}{Solution}
\begin{document}

\maketitle

\setcounter{secnumdepth}{0}



\section{Introduction}\label{introduction}

When children encounter people and animals in daily life, they
experience them through multiple sensory modalities simultaneously. When
playing with pets, for example, the child learns to associate the
physical features and actions of dogs and cows with the barking or
mooing sounds typically produced by these animals. Eventually, the child
will also link specific spoken words to each type of animal, learning
that they are associated with the object names dog or cow, as well as
with onomatopoeic words resembling their characteristic vocalizations,
such as woof-woof or mooo. Thus, just as familiar object names,
including animal names, are consistently associated with particular
kinds of animate and inanimate objects, animal vocalizations and
onomatopoeic words for vocalizations also provide consistent
associations between auditory stimuli and different types of animal. In
this study, we compare young children's efficiency in using these
different classes of sounds as cues to identifying particular animals.
In addition, we investigate whether children can use disambiguation
strategies, which have frequently been characterized as pragmatic or
lexically-specific in nature (Bloom, 2002; Diesendruck \& Markson, 2001;
Markman, 1991), in order to infer which of two animals is associated
with a novel vocalization.

The question of whether words are a special kind of stimulus for infants
is not new. Several studies have found advantages for speech sounds over
tones in object individuation and categorization in young infants
(Fulkerson \& Waxman, 2007; Xu, 2002). Focusing on associations between
objects and sounds, objects and tones, or objects and gestures, several
studies found that younger infants accept several different forms as
potential object labels, but that older infants seem to be more
discriminating and favor words (Namy \& Waxman, 1998; Woodward \& Hoyne,
1999). A different line of research found that infants prefer to hear
spoken words over some nonlinguistic analogues (Vouloumanos \& Werker,
2004, 2007a, 2007b), and that the neonate brain responds differently to
speech as compared to backwards speech (Pena et al., 2003). These
studies found advantages for speech over nonlinguistic analogues in
categorization, individuation, crossmodal association, and speech
preferences. However, they all focused on arbitrary nonlinguistic cues
that are not consistently associated to objects in children's everyday
environments.

Other research has approached the question of whether speech is special
from a different perspective, comparing how children process spoken
words as compared to non-arbitrary environmental sounds, such as animal
vocalizations (e.g., cat meowing) or the sounds produced by inanimate
objects (e.g., car starting). Studies with adults found similarities and
differences in both behavioral and neural responses to cross-modal
semantic associations between words and environmental sounds. In a
picture detection task, Chen and Spence (2011) found priming from
environmental sounds but not from words. These authors propose that
recognition of environmental sounds is faster because words must also be
processed at a lexical stage, while environmental sounds activate
semantic representations directly. In contrast, in a task in which
participants had to decide whether a sound and picture matched, Lupyan
and Thompson-Schill (2012) found advantages for words as compared to
environmental sounds. This finding was interpreted as evidence that
words activate conceptual information more quickly and accurately and in
a more categorical way than nonverbal sounds.

In an ERP imaging study with adults, Cummings, Ceponiene, Koyama,
Saygin, Townsend, and Dick (2006) looked at semantic integration of
verbal and non-verbal sounds and objects, and found that largely
overlapping neural networks processed verbal and non-verbal meaningful
sounds. In another study focusing on three different sound types that
varied in arbitrariness, Hashimoto et al. (2006) found different neural
mechanisms for the processing of animal names and vocalizations, with
onomatopoeic words activating both areas. Because research on
environmental sounds is still just beginning, it is hard to reconcile
these somewhat discrepant findings. Variations in tasks or timing of
stimuli can influence results, and different theoretical commitments can
lead to different interpretations. For example, environmental sounds are
often treated as encompassing both the sounds of living and man-made
objects (e.g.~cow mooing, bell ringing) despite evidence that these
sounds are treated differently by the adult brain (Murray, Camen,
Andino, Bovet, \& Clarke, 2006). But this line of research provides
promising new ways to examine the perennial question of whether language
emerges from the interaction of domain-general cognitive processes or
domain-specific mechanism (Bates \& MacWhinney, 1989). For example,
recent research comparing processing of speech and non-speech sounds is
leading to new insights relevant to autism, developmental language
impairment, and cochlear implants (Cummings \& Ceponiene, 2010;
McCleery, Ceponiene, Burner, Townsend, Kinnear, Schreibman, 2010).

From a developmental perspective, it is also important to understand how
children process words and non-arbitrary nonlinguistic sounds. Very few
studies have looked at this question. In a study using a
preferential-looking paradigm, Cummings, Saygin, Bates, and Dick (2009)
found that 15- and 25-mo-olds can use words and environmental sounds to
guide their attention to familiar objects, improving as they get older.
Vouloumanos, Druhen, Hauser, \& Huizink (2009) found that 5-month-olds
can match some animals to the vocalizations they produce. And studies
with children with autism and developmental language impairment found
more severe deficits for the processing of words than environmental
sounds (McCleery et al., 2010; Cummings \& Ceponiene, 2010).

We build on these earlier studies using the looking-while-listening
paradigm, which has been widely used to assess real-time interpretation
of spoken words by infants and young children (Fernald, Pinto, Swingley,
Weinberg, \& McRoberts, 1998; Fernald, Zangl, Portillo, \& Marchman,
2008). One major goal in this research is to investigate how
two-year-old children process different types of auditory stimuli that
are consistently associated with familiar animals, but that vary in
level of arbitrariness. First we ask whether 32-month-olds can use
onomatopoeic sounds (e.g.~bow-wow), and animal vocalizations (e.g., dog
barking) to identify familiar animals, as well as familiar animal names
(e.g., dog). By using real-time processing measures, we can also
determine whether these three sounds are equally effective as acoustic
cues in guiding children's attention to a particular animal in the
visual scene. The use of looking to visual stimuli, rather than
object-choice responses, reduce the task demands of procedures requiring
more complex responses such as reaching or pointing, and yield
continuous rather than categorical measures of attention on every trial,
capturing differences in processing that might not be detected by
offline tasks that rely on categorical responses (Fernald et al., 1998;
Bion, Borovsky, and Fernald, 2013).

A second major goal in this research is to investigate how young
children learn to link nonlinguistic sounds to animate objects. Do
two-year-olds make similar inferences when mapping a novel word and a
novel vocalization to an unfamiliar animal? Typically, word learning is
portrayed as an intractable challenge, while associating animals with
the sounds they produce might appear trivial. The acoustic structure of
vocalizations is influenced by the size and shape of the vocal tract and
other physical features, linking sounds to their source in a
non-arbitrary way. And the fact that many animal vocalizations are
accompanied by synchronous physical movements might provide children
with additional non-arbitrary cues to the source of the sound. Even in
the absence of additional visual cues, it is often possible to pinpoint
the source of a sound with reasonable accuracy. In contrast, because the
acoustic structure of a word is in most cases arbitrary with respect to
potential referents, and it is produced by a speaker and not by the
object itself, learning to associate speech sounds with objects is often
characterized as a complex problem of induction (Markman, 1991).

To solve the word-learning puzzle, children are said to be equipped with
constraints on the possible meanings of words. The most widely studied
of these constraints rules that each object must have only one name
(Markman, 1991). Evidence for this default assumption comes from
disambiguation tasks in which children hear a novel label in the
presence of a novel object and one or more familiar objects. In these
situations, children tend to select the novel object as the referent for
the novel word, presumably because the familiar objects already have
names associated with them. Debate about the origins, scope, and
generality of the Mutual Exclusivity (ME) constraint has focused on
whether this response bias provides evidence for a lexical constraint,
or whether it results instead from inferences about speakers'
communicative intent. Lexical accounts characterize ME as a
\enquote{domain specific mechanism specific to word leaning} (de
Marchena, Eigsti, Worek, Ono, \& Snedeker, 2011), which
\enquote{predicts disambiguation only within the domain of word learning
(i.e., it is domain-specific)} (Scofield \& Behrend, 2003). Pragmatic
accounts propose that ME extends to communicative acts more broadly,
reflecting assumptions that speakers are cooperative and should use
conventional names to refer to familiar objects (Bloom, 2002; Clark,
1990). A third possibility is that the bias toward one-to-one mappings
observed in word learning and other communicative domains reflect
general biases to find simple regularities in complex domains, a
perspective embraced by recent computational approaches to word learning
(Frank, Goodman, \& Tenenbaum, 2009; McMurray, Horst, \& Samuelson,
2013; Regier et al., 2003).

Thus, lexical and pragmatic accounts of the scope of the ME constraint
assert that one-to-one biases are either unique to word learning or that
they generalize to communicative acts more broadly, while domain-general
accounts predict that they would apply to any domain in which consistent
one-to-one mappings are observed.

To explore the possibility that one-to-one biases in sound-object
mappings are not limited to interpreting communicative acts, we
investigated whether children would show responses comparable to the
\enquote{mutual exclusivity} bias in a domain that is neither linguistic
nor communicative for them, but in which consistent associations are
observed between objects and auditory cues. The question of interest was
whether children would show one-to-one biases in linking novel,
non-speech vocalizations to unfamiliar animals, similar to their biases
in word learning contexts. When presented with paired pictures of a
familiar animal (e.g., dog) and an unfamiliar animal (e.g., aardvark) in
a disambiguation task, will 2-year-olds orient to the novel animal not
only when they hear a novel animal name, but also when they hear a novel
animal vocalization?

\section{Experiment 1}\label{experiment-1}

This first experiment asks two questions. First, we ask whether
32-month-olds can use familiar animal names (e.g., dog), onomatopoeic
words (e.g.~bow-wow), and animal vocalizations (e.g., dog barking) to
identify familiar animals. These three sound types differ in
arbitrariness within a continuum, with speech as the most arbitrary and
vocalizations as the least arbitrary sound. We ask whether they are
equally effective as acoustic cues in guiding children's attention to
animals in a visual scene. Children will hear a sound cue while looking
at the picture of two familiar animals, one that matches and one that
does not match the cue. We compare children's proportion of looking to
the matching animal when hearing the target sound. One of three patterns
of result is most likely to emerge: The first is that children are
faster to identify animal names than onomatopoeic words, and faster to
identify onomatopoeic sounds than animal vocalizations. This pattern of
results could be predicted by computational models that see frequency as
a crucial determinant of speed of processing (e.g., McMurray et al,
2013) either at the token level (children in urban environments are
likely to hear the names of animals more often than their vocalizations)
or in total amount, resulting in more practice interpreting speech (high
SES children might hear thousands of words daily, but not nearly as many
animal sounds). It could also be predicted by developmental accounts
that see speech as having a privileged status in early development
either in getting children's attention (Vouloumanos \& Werker, 2004,
2007a, 2007b), or due to the fact that it refers to objects (Waxman et
al., 2009). The second possible pattern of result is that children are
faster to identify animal vocalizations than onomatopoeic sounds, and
faster to identify onomatopoeic sounds than animal names. This pattern
of results could be predicted by accounts that propose that
non-arbitrary sounds link directly to semantic representations, while
words first connect to lexical representations before reaching semantics
(Chen \& Spence, 2011). The fact that children have experience
interpreting environmental sounds (e.g., balls bouncing, things falling)
before learning to interpret speech referentially could also predict an
advantage for environmental sounds. A third possibility is that children
are equally efficient in exploiting these three sound types to guide
their attention to a familiar animal. This pattern of results would
parallel that of previous studies that failed to find differences in the
processing of environmental sounds and words (Cummings et al., 2009).

Our second question is whether two-year-olds make similar inferences
when mapping a novel name and a novel animal vocalization to an
unfamiliar animal. In a paradigm similar to the one explained above,
children will hear a novel animal name or novel animal vocalization
(instead of a familiar one), while looking at the picture of a familiar
and a novel animal (instead of two familiar objects). We compare
children's proportion of looking to the novel animal when hearing one of
these two sound cues. Considering dozens of studies on children's
disambiguation biases, it is safe to assume that children will look at a
novel animal when hearing a novel name, thus one of two patterns of
results is most likely to emerge: The first is that children look at a
novel animal when hearing a novel name, but are at chance or
substantially less accurate when hearing a novel animal vocalization.
This pattern of result would be compatible with lexical accounts that
predict disambiguation only within the domain of word learning, or by
pragmatic accounts that predict disambiguation only within communicative
contexts. The second pattern of findings is that children look at a
novel animal when hearing a novel name and when hearing a novel animal
vocalization, with performance indistinguishable or comparable between
these two conditions. This pattern of results would be compatible with
accounts that propose that disambiguation biases emerge from
domain-general learning mechanisms that look for regularities in complex
domains.

\subsection{Method}\label{method}

\subsubsection{Participants}\label{participants}

Participants were 21 32-month-old children (M=31.8 months; range =
30.2-34.3), 10 girls. All were reported by parents to be typically
developing and from families where English was the dominant language.
Two participants were excluded due to fussiness. Children were from
mid/high-SES families, with average maternal education of 16 years
(i.e., college degree, range=12-18 years of education).

\begin{figure}
\centering
\includegraphics{anime_manuscript_files/figure-latex/unnamed-chunk-1-1.pdf}
\caption{\label{fig:unnamed-chunk-1}Trial types in Experiments 1 organized
by type of cue: Familiar vs.~Novel. The target animal for each trial
type is on the left.}
\end{figure}

\subsubsection{Visual stimuli}\label{visual-stimuli}

The visual stimuli included pictures of four Familiar animals (horse,
dog, cow, sheep) and two Novel animals (pangolin, tapir). According to
parental report, the familiar animals were known by all children.
Parents also reported that the novel animals were completely unfamiliar
to the children. Each animal picture was centered on a grey background
in a 640 x 480 pixel space

\subsubsection{Auditory stimuli}\label{auditory-stimuli}

The auditory stimuli consisted of sounds that were either Familiar or
Novel to 32-month-olds. Figure 1 serves as a guide to the different
sound types. The Familiar sounds were used in Familiar Trials, and
consisted of one of three different sounds: names (horse, dog, cow and
sheep), onomatopoeic words (neigh, woof-woof, moo and baa), and
vocalizations (horse neighing, dog barking, cow mooing and sheep
baaing). The Novel sounds were used in Disambiguation Trials, and
consisted of one of two types of sounds: names (capa, nadu) and
vocalizations (rhino grunting, gorilla snorting).

Trials in which the auditory cue was a familiar or novel animal name
(e.g., Where's the dog?) or a familiar or novel lexical sound (Which one
goes woof-woof?) began with a brief carrier frame. The duration of the
target cue was 810 ms for lexical sounds and 750 ms for animal names.
The intensity of the phrases was normalized using Praat speech analysis
software (Boersma, 2002).

Trials with familiar or novel animal vocalizations began with a single
word, used to draw children's attention (e.g., Look! \enquote{dog
barking}). Familiar animal vocalizations were selected based on
prototypicality. After selecting at least three vocalizations for each
familiar animal, the authors voted on the one that we thought would be
most easily recognized by children. Choosing the novel animal
vocalizations was more challenging. A group of research assistants
selected from different websites several vocalizations that they judged
as unfamiliar. From these vocalizations, we selected two (i.e., rhino
grunting and gorilla snorting) that we judged were equally likely to be
produced by the 6 familiar and 2 novel animals based on the their size
and vocal tract characteristics. These vocalizations were also maximally
distinct from each other and from the familiar animal vocalizations and
expected to be unfamiliar to children. We counterbalanced the
vocalizations that were paired with the two novel animals, in order to
control for the possibility that children judged one of the two novel
animals as more likely to produce one of the novel vocalization. All
children were reported by parents to have had no exposure to the novel
animal's natural vocalizations. The duration of the target animal
vocalizations was 2000 ms. More details about trial types and conditions
in Figure 1 will be given in the Procedure section.

\subsubsection{Familiarization books}\label{familiarization-books}

There were two main reasons for us to want to make sure that children
knew the familiar onomatopoeic words and animal vocalizations before we
administered our experiment. First, we wanted to make sure that any
differences we observed in children's performance within Familiar-Animal
Trials was due to processing speed and could not be explained by the
fact that children were not familiar with one of the sound types. The
looking-while-listening procedure has been shown to capture differences
in processing efficiency even when words are considered \enquote{known}
by offline reaching tasks or parental report (Fernald et al., 2008).
Second, we wanted to make sure that children knew the pairings between
familiar vocalizations and animals, a potential prerequisite for success
on Disambiguation trials. Since we were working with children from
mid/high-SES families growing up in an urban environment, we were
particularly concerned that they would not be familiar with many animal
vocalizations or onomatopoeic words. To ensure that all children had at
least some experience with the familiar animals and familiar auditory
cues used in our study, we gave two children's books to parents, both
titled Sounds on the Farm, a week before their visit. Parents were
instructed to share each book with their child for 5 to 10 min on at
least three days prior to the experiment. The first book consisted of
colorful pictures of each familiar animal and text designed to prompt
parents to produce each animal's lexical sound (e.g., Wow, look at all
those cows! This cow says moo, moo!). To give children exposure to the
natural animal vocalizations, we used a Hear and There book, which
contained buttons that children could press to hear the actual noise
that each animal produces.

\subsubsection{Procedure}\label{procedure}

Since we were interested in detecting differences in processing between
sounds that we expected to be familiar to children, we choose to access
speed and accuracy in identifying the correct target picture with the
looking-while-listening (LWL) procedure (see Fernald, et al, 2008).
Previous studies have shown that even when objects are reported by
parents as familiar to their children, or when children are at ceiling
in offline reaching tasks, these real-time processing measures can
capture meaningful differences in processing. These differences
correlate to properties of the sound stimuli (e.g., word-frequency) and
different aspects of the child's experience (e.g., their age,
socioeconomic status, amount of parental talk). Looking-time measures
have also been used in Disambiguation tasks with children from different
ages, capturing differences in accuracy that relate to children's age
and vocabulary size (Bion et al., 2013).

On each trial, a pair of pictures was presented on the screen for
approximately 4 s, with the auditory stimuli starting after 2 s,
followed by 1 s of silence. As seen in Figure 1, we have two main trials
types, Familiar Trials and Disambiguation trials, paralleling our
original two research questions on children's processing of familiar and
novel auditory cues.

Within the Familiar Trials, we have three different sub-trials: name,
onomatopoeic word, and vocalization. On 8 Name trials, each familiar
animal served as the target twice and was paired once with another
familiar animal and once with a novel animal. On 8 Onomatopoeic-word
trials, each familiar animal served as the target twice. On 16
Vocalization trials, each familiar animal served as the target four
times, paired twice with another familiar animal and twice with a novel
animal. These three familiar sound types should allow us to answer our
first research question, asking whether names, onomatopoeic words, or
animal vocalizations, are equally effective as acoustic cues in guiding
children's attention to animals in a visual scene.

Within the Disambiguation Trials, we have two different sub-trials:
name, and vocalizations. On 6 Name trials, each novel animal was labeled
three times with a novel animal name (i.e., capa, nadu), always paired
with a familiar animal. On 8 Vocalization trials, each novel animal
vocalization served as the target four times and was paired with each
familiar animal once. These two sound types should allow us to answer
our second research question, asking whether two-year-olds make similar
inferences when mapping a novel name and a novel animal vocalization to
an unfamiliar animal.

These different trial types were administered in two different visits.
The Familiar and Disambiguation Trials with animal names and
onomatopoeic words were administered during children's first visit. The
Familiar and Disambiguation Trials with the animal vocalizations were
administered during their second visit. We administered the animal
vocalizations on the second visit to allow children to become familiar
with the procedure and to give parents additional time to use the
familiarization books with the vocalizations with their children. During
each visit, five Filler trials were interspersed throughout to add
variety and maintain children's attention. Pairings of the novel animal
and name, and side of presentation of target animals, were
counterbalanced across participants. Caregivers wore darkened sunglasses
so that they could not see the pictures and influence infants' looking
throughout the 5-min procedure.

\subsubsection{Measures of processing
efficiency}\label{measures-of-processing-efficiency}

Participants' eye movements were video-recorded and coded with a
precision of 33 ms by observers who were blind to trial type. Inter- and
intra-observer reliability checks were conducted for all coders. For
25\% of the subjects, two measures of inter-observer reliability were
assessed. The first was the proportion of frames (33-ms units) on each
trial on which two coders agreed. In this case, agreement was 98\%.
However, because this analysis included many frames on which the child
was maintaining fixation on one picture, we also calculated a more
stringent test of reliability. This second measure focused only on
shifts in gaze, ignoring steady-state fixations in each trial on which
agreement was inevitably high. By this more conservative measure, coders
agreed within one frame on 94\% of all shifts.

\textbf{Accuracy:} On those trials in which the infant was fixating a
picture at the onset of the speech stimulus, accuracy was computed by
dividing the time looking to the target object by the time looking to
both target and distracter, from 300 to 4300 ms from the onset of the
target word. Accuracy before 300 ms was not included because shifts to
the target occurring in this window had presumably been initiated before
the onset of the noun. This analyses window included the entire duration
of the trial, and it was longer than that of studies with familiar words
(Fernald et al., 1996) because of the longer duration of the animal
vocalizations (2 s.) and because of the introduction of novel auditory
cues. A single analyses window was used for all trial types for
consistency, and the entire duration of the trial was used in order to
avoid arbitrary decisions. Mean accuracy was then computed for each
participant on each trial type. Similar results are found when using a
shorter analysis window.

\textbf{Reaction time:} We calculated reaction time (RT) on those trials
on which participants were looking at the distractor animal at the
beginning of the sound. RT on each trial was the latency of the first
shift to the correct animal within a 300- to 1,800-ms window from sound
onset, as typically done in studies using this procedure (Fernald et
al., 1998).

\subsection{Results and discussion}\label{results-and-discussion}

\subsubsection{Familiar Trials: Using familiar animal names, lexicalized
sounds, and animal vocalization to identify familiar
animals:}\label{familiar-trials-using-familiar-animal-names-lexicalized-sounds-and-animal-vocalization-to-identify-familiar-animals}

Our first question is whether a familiar animal name, onomatopoeic word,
and animal vocalization is a particularly effective cue in guiding
children's attention to an animal in their environment. To begin
exploring this question, Figure 2 shows children's looking behavior on
the LWL procedure (Fernald et al., 1998). In order to capture children's
speed of processing, we only show children's responses on trials in
which they start looking at the wrong animal. From sound onset onward,
we show the mean proportion of trials in which children were looking at
the correct picture, every 33ms, with different lines representing
children's responses on Name trials (black), Onomatopoeic-word trials
(dark grey), and Vocalization trials (light grey). The y-axis shows the
mean proportion of trials on which children are looking at the correct
animal. The x-axis represents time from sound onset in milliseconds. As
can be seen in Figure 1, at 750ms from sound onset, there is already a
substantial difference in the mean proportion of trials in which
children are looking at the correct animal depending on whether they
heard an animal name or an animal vocalization. Children are looking at
the correct animal in a greater proportion of trials when they hear a
familiar animal name, as compared to when they hear an animal
vocalization, with performance on trials with onomatopoeic words
somewhere in between. This difference is robust and continues to
increase as the trial unfolds.

To assess these differences statistically, we computed children's
reaction time (RT) as the mean time it took them to shift to the correct
picture on trials in which they were looking at the wrong picture at
sound onset, again separately for the three trial types. We first run a
general model confirming that children's RT differed across the three
conditions (F(2, 36) = 5.732, p = 0.002). Next, we individually compared
children's performance: Children were faster to identify the target
animal when hearing its name (M = 541ms), as compared to its
vocalization (M = 840 ms, t (18) = 4.61, p \textless{} 0.001) or
lexicalized sound (M = 801 ms, t (18) = 3.40, p = 0.002), with the last
two conditions not differing from each other (t (18) = 0.47, p
\textgreater{} 0.6).

\begin{figure}
\centering
\includegraphics{anime_manuscript_files/figure-latex/unnamed-chunk-2-1.pdf}
\caption{\label{fig:unnamed-chunk-2}Time course of children's looking to the
target animal after hearing a familiar animal name, onomatopoeic word,
or animal vocalization. The graph shows children's responses for the
trials in which they were looking at the distracter animal at sound
onset. The x-axis shows time in milliseconds from sound onset. The
y-axis shows the mean proportion of trials in which children are looking
to the target animal. When hearing a familiar auditory cue, children
were faster to orient to the correct animal after hearing its name, than
when hearing the lexicalized sound or animal vocalization.}
\end{figure}

When looking at children's reaction time, we found a difference in speed
of processing for names, onomatopoeic words, and vocalization. Our next
step was to confirm that children could use these three trial types to
guide their attention to a familiar animal by comparing children's
accuracy against chance performance (50\% accuracy). When accuracy is
computed over a window from 300 to 4300 ms after the onset of the cue,
the three types of auditory cue were equally effective in guiding
children's attention to the target animal. Children looked to the
correct animal when hearing the animal name (M = 0.62, t (18) = 3.57, p
= 0.002), the lexicalized sound (M = 0.64, t (18) = 3.54, p = 0.002),
and the animal vocalization (M = 0.64, t (18) = 5.90, p \textless{}
0.001). A general linear model confirmed that there was no statistically
significant difference between children's accuracy in these three
conditions (F(2, 36) = .137, p = 0.873). These results show that the
animal vocalization can be as good a cue to identify familiar animals as
their names or lexicalized sounds if one gives children enough time to
process the stimuli.

\subsubsection{Disambiguation Trials: Using novel animal names and
animal vocalization to disambiguate novel
animals}\label{disambiguation-trials-using-novel-animal-names-and-animal-vocalization-to-disambiguate-novel-animals}

Our second question is whether children would orient to a novel animal
after hearing a novel animal name or vocalization. In order to evaluate
this research question, we focus on children's accuracy, comparing their
proportion of looking to the novel animal against chance performance. We
started by fitting a general linear model with familiarity (familiar or
novel sound) and type of sound (animal name or vocalization) as
within-subject variables predicting children's proportion of looks to
the novel animal. Figure 3 shows children's proportion of looking to the
novel animal in the different conditions. We found a main effect of
familiarity, meaning that children were much more likely to look at the
novel animal when hearing a novel sound than when hearing a familiar
sound (F(1, 18) = 38.04, p \textless{} 0.001 ). We found no effect of
type of sound, meaning that children's accuracy was indistinguishable
when they heard a name or a vocalization (F(1, 18) = 0.120, p = 0.733).
We also found no interaction between familiarity and type of cue (F(1,
18) = 0.045, p = 0.834).

\begin{figure}
\centering
\includegraphics{anime_manuscript_files/figure-latex/unnamed-chunk-3-1.pdf}
\caption{\label{fig:unnamed-chunk-3}Accuracy of responses to novel and
familiar auditory cues. When hearing novel animal names or animal
vocalizations, children reliably looked at the novel animal. When
hearing a familiar animal name or animal vocalization, children reliably
looked at the familiar animal. The different auditory cues were equally
effective in guiding children's attention to the target.}
\end{figure}

When performance is compared against chance level of 0.5, children
reliably looked to the novel animal when hearing the novel animal name
(M = 0.62, t(18) = 3.101, p = 0.006) or novel animal vocalization (M =
0.63, t(18) = 6.25, p \textless{} 0.001). Children reliably looked at
the familiar animal when hearing the familiar animal name (M = 0.65,
t(18) = 3.637, p = 0.001) or familiar animal vocalization (M = 0.64,
t(18) = 4.43, p \textless{} 0.001).

Therefore, children seem to have one-to-one biases for the vocalizations
that animals produce already at 30 months of age, the earliest age at
which the disambiguation effect has been observed in a domain other than
word learning. There were no significant differences between children's
reaction time for novel animal names or vocalizations.

\section{Experiment 2}\label{experiment-2}

An issue that has received much attention in recent years concerns the
relation between children's referent selection and retention abilities.
While earlier studies tended to conflate disambiguation strategies and
children's word learning, more recent studies suggest that these two
abilities should not be conflated (Bion, Borovsky, \& Fernald, 2013;
Horst, McMurray, \& Samuelson, 2006).

Horst and Samuelson (2008) examined both referent selection and
retention in four experiments with 2-year-olds. When children were shown
a novel object among familiar objects, they selected the novel object
when hearing a novel label, as found in previous studies. But
surprisingly, on retention trials 5 min later, these children showed no
evidence of remembering the names of the novel objects they had
previously identified. Using a looking-time task, Bion et al. (2013)
replicated these findings in a study with 18-, 24-, and 30-month-old
infants using looking time measures of performance.

Experiment 2 asks whether children can retain the link created through
disambiguation between a novel animal and a novel animal vocalization.
In addition, we aim at replicating the findings from Experiment 1,
showing that children can identify familiar animals based on the
vocalizations they produce, and use novel vocalizations to disambiguate
novel animals. Our prediction is that children will succeed in
disambiguation trials, but will show marginal or no retention on
subsequent disambiguation trials, paralleling the findings of earlier
studies with linguistic stimuli (Bion et al., 2013).

\subsection{Method}\label{method-1}

\subsubsection{Participants}\label{participants-1}

Participants were 22 31-month-old children (M=31.1 months; range =
27.4-32.5), 12 girls. All were reported by parents to be typically
developing and from families where English was the dominant language.
Children were from mid/high SES families, with an average of 17 years of
maternal education (range = 14-18 years of education).

\subsubsection{Visual stimuli}\label{visual-stimuli-1}

The visual stimuli were the same as in Experiment 1, except for the
novel animals (aardvark and capybara), which replaced the novel animals
(pangolin and tapir) used in Experiment 1 (see animals in Figure 4). We
decided to change the novel animals in order to confirm that our results
were not restricted to the particular stimuli set in Experiment 1. All
children were reported by parents to have had no exposure to the novel
animals.

\subsubsection{Auditory stimuli}\label{auditory-stimuli-1}

The auditory stimuli consisted of the same familiar and novel animal
vocalizations as in Experiment 1.

\begin{figure}
\centering
\includegraphics{anime_manuscript_files/figure-latex/stimuli_e2-1.pdf}
\caption{(\#fig:stimuli\_e2)Trial types in Experiments 4 organized by
type of trial. Children hear familiar and novel vocalizations. The
target animal for each trial type is on the left.}
\end{figure}

\subsubsection{Familiarization books}\label{familiarization-books-1}

As in Experiment 1, we sent home a children's book to ensure that all
participants had at least some exposure to the familiar animals and
auditory cues. Since, in Experiment 2, we were interested in the natural
animal vocalizations and not the names/lexical sounds, only the Hear and
ThereTM Sounds on the Farm book was used. Instructions given to the
parents were the same as in Experiment 1, and the book was sent home a
week before the visit.

\subsubsection{Procedure}\label{procedure-1}

Experiment 2 consisted of one visit. Each child saw 30 trials,
consisting of three trial types (Figure 4). The 16 Familiar trials and 8
Disambiguation trials were identical in structure to the Vocalization
trials Experiment 1. In addition, on 6 Retention trials, the two novel
animals were presented side by side, with each serving as the target
three times. The same coding and speed/accuracy measures were used as in
Experiment 1.

\subsection{Results and discussion}\label{results-and-discussion-1}

\subsubsection{Retention of the link between a novel animal and a novel
vocalization:}\label{retention-of-the-link-between-a-novel-animal-and-a-novel-vocalization}

Figure 5 presents children's proportion looking to the target animal
after hearing a familiar or a novel animal vocalization over a window
from 300 to 4300 ms after the onset of the vocalization. We start by
fitting a general linear model with the three conditions as
within-subject factors. The main effect of condition confirmed that
performance on retention trials was lower than in the other conditions
(F(2, 22) = 3.890, p = 0.028). When next compared accuracy against
chance level of 0.5. When children heard a familiar animal vocalization,
they oriented to the target familiar animal (M = 0.62, t (21) = 5.47, p
\textless{} 0.001). When children heard a novel animal vocalization,
they looked at a novel animal instead (M = 0.63, t (21) = 3.76, p
\textless{} 0.001). These results replicate the main finding from
Experiment 1.

\begin{figure}
\centering
\includegraphics{anime_manuscript_files/figure-latex/prop_look_e2-1.pdf}
\caption{(\#fig:prop\_look\_e2)Accuracy of responses to familiar and
novel animal vocalizations. Children reliably looked to the target
animal after hearing a familiar animal vocalization and a novel animal
vocalization. Children did not show evidence of retaining the mapping
between a novel animal and a novel vocalization.}
\end{figure}

Can children remember the association between the novel animal and the
novel vocalization, when this association is created through a
disambiguation strategy? In order to test this hypothesis, two novel
animals were paired with each other, as children heard the animal
vocalization previously associated with one of the two animals.Children
were marginally successful in retaining the link between the novel
animal and the novel vocalization (Figure 4, M = 0.56, t (21) = 1.79, p
= 0.089).

Three findings emerged from this analyses of accuracy, the first two
replicated findings from Experiment 1: First, children oriented to a
familiar animal after hearing a familiar animal vocalization. Second,
children oriented to a novel animal after hearing a novel animal
vocalization. In addition, we also found that children were marginally
successful in retaining the link between a novel animal vocalization and
a novel animal. These results and their implications will be discussed
in more details in the following section.

\section{General Discussion.}\label{general-discussion.}

Three main findings emerged in this research. The first finding was that
30-month-olds responded fastest to the familiar animal name and slowest
to the familiar animal vocalization, with onomatopoeic sounds somewhere
in between. Yet, children could identify the familiar animals after
hearing any of on these three sound types. The second finding was that
children showed disambiguation biases for the types of vocalizations
that animals produce, similar to their biases in word learning. The
third finding was that these biases do not necessarily lead to learning,
as children were only marginally successful in retaining the link
between novel animals and their vocalizations, paralleling the findings
of recent word-learning studies (see McMurray, Horst, et al., 2009).

In our study, we found an advantage for words over other meaningful
sounds. One of the reasons for this advantage for words might be due to
differences in the processing of these two sound types. Some language
development theories argue that words are special because they refer to
objects in the world (Waxman \& Gelman, 2009), and other theories argue
that words are special because they activate conceptual information more
quickly and accurately and in a more categorical way than nonverbal
sounds. And it is also possible that words and nonverbal sounds might be
processed by different brain regions, with words being more rapidly
accessed. A second explanation for the advantage for words might be due
to differences in frequency. At least in our particular sample, it is
save to assume that children have heard the word cow many more times
than they have heard an actual cow mooing. Frequency effects have been
robustly demonstrated in the processing of words, with adults being
faster to recognize words that they hear more frequently (Dahan,
Magnuson, \& Tanenhaus, 2001). And third, words are very effective at
presenting a lot of information in a short period of time. When a dog
and a sheep are paired with each other, the first phoneme of the words
is enough to determine the animal that is likely to be talked about
next.

Little is known about children's and adults' processing of onomatopoeic
sounds. Hashimoto et al. (Hashimoto et al., 2006) compared brain
responses to nouns, animal sounds, and onomatopoetic sounds, and found
that onomatopoeic sounds were processed by extensive brain regions
involved in the processing of both verbal and nonverbal sounds. Cummings
et al. argues that onomatopoeic sounds might provide young children with
information about intermodal associations, bridging their understanding
of non-arbitrary environmental sounds and arbitrary word-object
associations (Cummings et al., 2009). Fernald and Morikawa (1993)
reported that 52\% of Japanese mothers used onomatopoeic sounds to label
target objects, while only 1 in 30 American mothers did so.

Our second finding was that children looked at a novel animal when
hearing a novel animal vocalization, with accuracy comparable to their
disambiguation of novel animal names. In the book How Children Learn the
Meaning of Words, Paul Bloom discusses different theories explaining
children's disambiguation biases. These biases could be a specifically
lexical phenomenon that applies only to words (i.e., lexical account), a
product of children's theory of mind restricted to communicative
situations (i.e., pragmatic account), or a special case of a general
principle of learning that exaggerates regularities across domains
(i.e., domain-general account). As a possible way to decide among these
three theories, he suggests that it would be interesting to know if
children show disambiguation biases when learning the sounds that
different animals make (p., 69, Bloom, 2002).

Previous studies contrasted lexical-specific and pragmatic accounts,
with little attention to domain-general explanations. Diesendruk and
Markson found that children expect speakers to use consistent facts to
refer to objects, and they select a novel object when hearing a novel
fact (Diesendruck \& Markson, 2001). Yet, recent studies suggest that
different strategies might be used to make inferences about speakers'
communicative intent and the meaning of a novel word. Autistic children
who are known to have pragmatic deficits show disambiguation biases and
select a novel word when hearing a novel object (Preissler \& Carey,
2005). Disambiguation biases for words are correlated with vocabulary,
and disambiguation biases for facts are correlated with social-pragmatic
skills (de Marchena et al., 2011). As these authors acknowledge, these
findings suggest that disambiguation biases for words might not be
motivated uniquely by pragmatic inferences, but they do not rule out
domain-general accounts.

Few studies looked at disambiguation biases in non-linguistic domains.
Three-year-olds expect faces to map to individual voices (Moher,
Feigenson, \& Halberda, 2010), showing that one-to-one biases might
extend to other communicative domains. In contrast, studies with adults
found an advantage for words over non-linguistic stimuli in a task that
could benefit from disambiguation biases (Yoshida, Rhemtulla, \&
Vouloumanos, 2012). The findings with adults do not rule out a
domain-general account, since participants had no reason to expect that
the mapping between random non-linguistic sounds and objects should be
mutually exclusive. As pointed out by the authors, perhaps similar
strategies might be applied to nonlinguistic sounds if they were
meaningfully related to the object.

Our study was the first to demonstrate that young children show
disambiguation biases in a nonlinguistic and non-communicative domain.
This is also the youngest age at which disambiguation biases were shown
in a domain other than word learning. These findings seem to favor
domain-general accounts that see disambiguation biases as the natural
consequence of a system that attempts to find regularities in complex
learning tasks that involve consistent mappings. Previous connectionist
and Bayesian models of word learning showed that disambiguation biases
emerge as children are exposed to consistent mappings between words and
objects, without the need of built-in constraints on the meaning of
words (Frank et al., 2009; McMurray, Horst, \& Samuelson, 2013; Regier,
2003; Yu \& Smith, 2007). In principle, these same biases would emerge
if these models attempted to map animal vocalizations to animals and
consistent co-occurrences were present in the environment.

The question that remains is whether the finding from our studies and
from previous studies could be explained by a single disambiguation
mechanism. We showed that children can disambiguate stimuli other than
words and facts, suggesting at least the existence of a domain general
mechanism that leads to disambiguation. A preference for parsimony
provides some week evidence in favor of a single mechanism. But as
pointed out by recent computational work, it is possible that different
mechanisms jointly contribute to disambiguation behavior, explaining
findings across different populations and contexts (Lewis \& Frank,
2013). It is possible therefore that the same behavior - selecting a
novel object when hearing a novel auditory stimulus - might result from
different computational mechanisms or motivations depending on the task
at hand, children's age, or the particular stimuli set and assumptions
about the people involved in the interaction. That is, children could
use a domain general mechanism to learn about novel animal
vocalizations, and they could use a lexical and pragmatic constraint to
learn about novel words.

Our third finding was that one-to-one biases for animal vocalizations do
not necessarily lead to retention of the link between a novel animal and
a novel vocalization. Importantly, this finding supports the prediction
of recent cross-situational models of early word learning (Horst et al.,
2006; McMurray, Horst, \& Samuelson, 2013). McMurray and colleagues
propose that referent-selection requires that children give their best
guess in a specific ambiguous situation, but learning operates over a
much longer time scale. Although disambiguation can be viewed as the
product of learning that has occurred up to that point, for younger
children it does not necessarily result in learning. These claims are
corroborated by studies on early word learning using online and offline
measures of retention (Horst, et al., 2006, Bion et al., 2013).

The findings of this study add to a recent body of work that encourages
us to think differently about disambiguation biases. These studies have
emphasized the role of experience in the emergence of disambiguation
biases, showing that the tendency to select a novel object when hearing
a novel word is not robustly present across populations. For example,
bilingual children, children from lower socioeconomic status, children
who receive less language input, and children with less structured
vocabularies or smaller vocabulary sizes, take longer to show evidence
of disambiguation biases (Bion, Borovsky, \& Fernald, 2013; Weisleder,
Hurtado, \& Fernald, in preparation; Yurovsky et al., 2012). Other
studies have problematized the relation between disambiguation biases
and word learning, showing that success in referent selection does not
necessarily mean that the link between the novel word and novel object
will be retained (Bion, Borovsky, \& Fernald, 2013; Horst, McMurray, \&
Samuelson, 2006; McMurray, Horst, \& Samuelson, 2012). And the current
study adds to recent studies taking a fresh look at an old question: the
scope of disambiguation biases (Suanda \& Namy, 2012, 2013). Taken
together, these recent studies challenge some widespread assumptions
about the emergence, importance, and scope of a behavior often
characterized as innate and universal.

Children's learning about objects in their environment involves more
than learning their names. Before object names are learned, sounds and
actions might form the basis on which objects are conceptualized
(Nelson, 1973). For example, children might see barking as a defining
feature of dogs, and may say bow-wow in response to the picture of a
dog, even before they learn the animal name (Nelson, 1973). Learning the
meaning of an object therefore requires learning several cross-modal
associations, including learning the object's texture, smell, as well as
its sounds and names. Children do not have explicit constraints that
freshly baked cookies should have only one smell. Yet, they might
recognize and get excited about the familiar smell coming from the
kitchen, and might assume their mothers are baking something new when
smelling something unfamiliar.

Children use different types of knowledge in order to make sense of a
constantly changing world. They might identify animal vocalizations
based on the shape of the vocal tract of the animal, its location and
size, and their previous knowledge about animal vocalizations.
Importantly, these cues normally converge in helping children identify
an animal in the environment. The same is true for their identification
of referents for words. Children can identify the referent for a word
based on semantics (Goodman, McDonough, \& Brown, 1998),
cross-situational statistics (Goodman et al., 1998), syntax (Brown,
1957), and pragmatic and social cues (Baldwin, 1991, 1993), and -- why
not - disambiguation biases (Markman, 1991). As children grow older,
these different sources of information provide converging evidence that
a novel word should refer to a novel object. Children can rely on their
knowledge about the world, speakers, and on their previous experiences
with words in order to figure out what speakers are talking about -- a
task we continue to do throughout our lives when learning knew words and
interpreting complex sentences.

\newpage

\section{References}\label{references}

\setlength{\parindent}{-0.5in} \setlength{\leftskip}{0.5in}






\end{document}
